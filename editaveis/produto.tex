\chapter{Produtos, Atividades e Cronograma}

\section{Estrutura Analítica do Projeto}

\FloatBarrier
\begin{figure}[!htbp]
		\centering
		\includegraphics[scale=0.40]{figuras/eap}
		\label{img:processo}
		\caption{Visão geral do Processo de Medição}
\end{figure}
\FloatBarrier

\section{Descrição das Atividades}

\FloatBarrier
\begin{table}[!htbp]
\centering
\caption{\textbf{Estabelecer e manter compromisso com o processo de medição}}
\label{my-label}
\begin{tabular}{|c|c|c|}
\hline
Atividade                & Descrição                                                                                                                                                               & Responsáveis \\ \hline
\begin{tabular}[c]{@{}c@{}} Definir Área \\ de Melhoria\end{tabular} & \begin{tabular}[c]{@{}c@{}}Esta atividade consiste em definir um projeto, bem \\ como a área a ser realizada o processo de medição\\ e em definir o escopo\end{tabular} & Todos        \\ \hline
\begin{tabular}[c]{@{}c@{}} Atribuir \\ Recursos\end{tabular}       & \begin{tabular}[c]{@{}c@{}}Atividade responsável por alocar recursos com \\ competênciapara a execução do processo\end{tabular}                                         & Todos        \\ \hline
\end{tabular}
\end{table}
\FloatBarrier

\FloatBarrier
\begin{table}[!htbp]
\centering
\caption{\textbf{Planejar o processo de medição}}
\label{my-label}
\begin{tabular}{|c|c|c|}
\hline
Atividade                                                           & Descrição                                                                                                                            & Responsáveis \\ \hline
\begin{tabular}[c]{@{}c@{}}Definir \\Medições\end{tabular}                                                  & \begin{tabular}[c]{@{}c@{}}Esta atividade consiste em definir as métricas\\ a serem coletadas durante o processo\end{tabular}        & Todos        \\ \hline
\begin{tabular}[c]{@{}c@{}}Definir Plano\\ de Medições\end{tabular} & \begin{tabular}[c]{@{}c@{}}Esta atividade consiste em estabelecer como será \\ executada a coleta e análise de métricas\end{tabular} & Todos        \\ \hline
\end{tabular}
\end{table}
\FloatBarrier

\FloatBarrier
\begin{table}[!htbp]
\centering
\caption{\textbf{Executar o processo de medição}}
\label{my-label}
\begin{tabular}{|c|c|c|}
\hline
Atividade                                                       & Descrição                                                                                                                                                                               & Responsáveis \\ \hline
\begin{tabular}[c]{@{}c@{}}Coletar \\ Dados\end{tabular}        & \begin{tabular}[c]{@{}c@{}}Atividade responsável pela coleta de métricas\\  previamente definidas\end{tabular}                                                                          & Todos        \\ \hline
\begin{tabular}[c]{@{}c@{}}Analisar\\ Dados\end{tabular}        & \begin{tabular}[c]{@{}c@{}}Esta atividade consiste na análise das  métricas \\ coletadas, a fim de que obtenha-se um \\ entendimento do impacto dessas métricas no projeto\end{tabular} & Todos        \\ \hline
\begin{tabular}[c]{@{}c@{}}Comunicar\\  Resultados\end{tabular} & \begin{tabular}[c]{@{}c@{}}Esta atividade consiste na descrição\\  da análise feita sobre as métricas\end{tabular}                                                                      & Todos        \\ \hline
\end{tabular}
\end{table}
\FloatBarrier

\FloatBarrier
\begin{table}[!htbp]
\centering
\caption{\textbf{Avaliar o processo de medição}}
\label{my-label}
\begin{tabular}{|c|c|c|}
\hline
Atividade                                                        & Descrição                                                                                                                                       & Responsáveis \\ \hline
\begin{tabular}[c]{@{}c@{}}Identificar\\  Melhorias\end{tabular} & \begin{tabular}[c]{@{}c@{}}Atividade responsável por identificar possíveis \\ melhorias a serem feitas dada a análise das métricas\end{tabular} & Todos        \\ \hline
\end{tabular}
\end{table}
\FloatBarrier

\subsection{Cronograma de Atividades}

O cronograma de todas as atividades está pautado nos prazos que devem ser seguidos para a implantação do processo de medição. Este cronograma pode sofrer alterações durante a execução do processo de medição, e deve ser validado com as partes interessadas no processo, tal como a empresa que utilizará o plano, e a equipe que irá implantar o plano.

Para todos os fins, as datas do cronograma devem ser seguidas por ambas as partes, tal instrumento deve se utilizado como mecanismo de gerenciamento do processo de medição como um todo.

\FloatBarrier
\begin{figure}[!htbp]
		\centering
		\includegraphics[scale=0.40]{figuras/cronograma}
		\label{img: cronograma do processo}
		\caption{Cronograma do Processo de Medição}
\end{figure}
\FloatBarrier

\section{Lista de Software}
A empresa Wavez utiliza as seguintes ferramentas para desenvolver o aplicativo MyFit:
\begin{itemize}
\item Linguagem Android
\item Controle de Versão Git
\item Slack 
\end{itemize}
Os integrantes da equipe de Medição e Análise utilizam como ferramentas, as seguintes:
\begin{itemize}
\item Controle de Versão Git
\item Slack 
\item Látex
\item Google Docs
\item Telegram
\end{itemize}