\chapter{Introdução}
\section{Contexto}
Este relatório contempla o processo de medição a ser aplicado na divisão de desenvolvimento \textit{mobile} da Wavez. Esta é uma empresa focada em \textit{marketing} digital, e estando inserida neste meio, consequentemente possui uma área voltada para o desenvolvimento de \textit{software}. Por ser uma empresa jovem, a mesma não possui um processo de desenvolvimento de \textit{software} bem definido, assim sendo também não possuem um processo de medição.

Em uma reunião com o CEO da empresa e alguns gerentes de projeto, foi observado que os maiores gargalos no desenvolvimento de \textit{software} estão relacionados à definição do escopo, estimativa de tempo e produtividade da equipe e com isso pudemos definir as áreas a serem atacadas. Neste relatório serão descritas as métricas que serão utilizadas, o plano e o processo de medição.
\section{Objetivo}
O objetivo deste trabalho é a criação e implantação de um processo de medições no projeto MyFit, em desenvolvimento na Wavez. No fim deste projeto é esperado que este processo de medição seja continuado e implantado nos demais projetos da empresa, a fim de melhorar todo o processo de desenvolvimento de \textit{software}.
\section{Justificativa}
Atualmente, os processos de medição são peças fundamentais em qualquer organização séria que tenha um processo de desenvolvimento de software. Isso se dá devido às rápidas e constantes mudanças que caracterizam esta tão importante área da sociedade, em que empresas que queiram se manter no mercado de forma competitiva devem optar por processos que tenham um alto nível de qualidade, produtividade e eficiência.

Um bom processo de medição auxilia a organização a tomar as melhores decisões, pois:
\begin{itemize}
\item Melhora o entendimento dos processos aplicados, dos produtos, recursos e ambientes
\item Mensura o andamento do projeto em relação ao planejado
\item Aponta ações a serem tomadas de acordo com informações coletadas e analisadas
\item Identifica gargalos no processo que possam vir a causar impedimentos ou queda na qualidade do produto final
\end{itemize}
