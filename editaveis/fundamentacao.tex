\chapter{Fundamentação Teórica}

O desenvolvimento do presente projeto de medição foi fundamentado inicialmente no conteúdo ministrado na disciplina de Medição e Análise, do curso de Engenharia de Software da Universidade de Brasília, baseado nas Métricas Fundamentais e Processos de Medição (ISO 15939).

Para o aprofundamento no contexto, foi estudado, além do processo de medição baseado no GQM (\cite{GQM2009}), artigos que pudessem dar uma visão mais contextualizada do processo a se aplicar na empresa Wavez, por vez que esta possui particularidades de relevância para se fazer o planejamento e levantamento das reais necessidades desta, como é previsto e tratado nos artigos \cite{franca1998mediccao}, \cite{anacleto2004metodo} e \cite{salviano2006proposta}, baseados em aplicações para processo de software de micro e pequenas empresas.

Também foi tomado como fundamentação o artigo \cite{schnaider2004abordagem} que aborda o precesso de medição e análise bem como ferramentas utilizadas para o apoio na elaboração de planos de medições para Organização e Projeto e na obtenção e fornecimento dos resultados das medições realizadas de acordo com as necessidades e objetivos dos Planos de Medição da
Organização e dos projetos.






\bibliographystyle{apa}
\bibliography{bibliografia}
 
 